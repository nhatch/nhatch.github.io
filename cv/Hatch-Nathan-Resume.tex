%%%%%%%%%%%%%%%%%%%%%%%%%%%%%%%%%%%%%%%%%
% Medium Length Professional CV
% LaTeX Template
% Version 2.0 (8/5/13)
%
% This template has been downloaded from:
% http://www.LaTeXTemplates.com
%
% Original author:
% Trey Hunner (http://www.treyhunner.com/)
%
% Important note:
% This template requires the resume.cls file to be in the same directory as the
% .tex file. The resume.cls file provides the resume style used for structuring the
% document.
%
%%%%%%%%%%%%%%%%%%%%%%%%%%%%%%%%%%%%%%%%%

\documentclass{resume} % Use the custom resume.cls style

\usepackage[left=0.75in,top=0.6in,right=0.75in,bottom=0.6in]{geometry} % Document margins

\name{Nathan T. Hatch} % Your name
\address{\href{https://nhatch.github.io}{https://nhatch.github.io} \\ Seattle, WA, USA \\
         \href{mailto:nhatch2REMOVE_THIS_PART@uw.edu}{nhatch2@uw.edu} \\ +1 970-297-8081 }

\begin{document}

\begin{rSection}{Robotic Systems Engineering Experience}

\begin{rSubsection}{Software Team Lead}{September 2021 - present}{RACER Project, U. of Washington}{Seattle, WA}
{\bf\small Develop perception, planning, and control software for $\sim$10 m/s autonomous navigation of a Polaris RZR in off-road terrain including dirt trails, steep hills, tall grass, bushes, trees, water, and rocks}
\item Brainstorm and prioritize big projects that might improve autonomous performance (e.g. variable dt)
\item Design and implement features for local planning and control (e.g. attitude costs)
\item Design and implement tools for efficient development and field tests (e.g. testing on rosbag replay)
\item Curate regression test cases and tune parameters to pass the tests
\item Prepare software for weekly field tests (code review, simulation validation, writing field test plan)
\item Attend field tests, direct experiments, drive chase vehicle, and assist with recovery after accidents
\item Analyze field test results to identify issues and propose solutions (e.g. open-loop execution at waypoints)
\item Manage 10-person team (maintain task tracking system, weekly group meeting, and several 1-1s)
\item Recruit and interview team members and maintain onboarding materials
\item Explain technical progress to stakeholders, including UW PIs and DARPA program manager
\item Handle software-related logistics (e.g. data offload, security, shipping computing equipment)
\end{rSubsection}

\begin{rSubsection}{System Engineer}{June 2019 - September 2021}{SARA Project, U. of Washington}{Seattle, WA}
\item Conducted weekly field experiments for perception, planning, and control of a Clearpath Warthog robot (a.k.a. Argo J5 XTR) outfitted with cameras and an Ouster OS2 LIDAR sensor
\item Handled physical and electrical hardware integration for new sensors
\item Sped up the LIDAR processing pipeline to 10Hz to support 3m/s vehicle velocities
\end{rSubsection}

\begin{rSubsection}{Software Subsystem Lead}{January 2020 - September 2021}{Husky Robotics Club, U. of Washington}{ Seattle, WA}
\item Wrote software for teleoperation and autonomous control of a student-designed and -built Mars rover
\item Recruited team members and delegated tasks to prepare for the University Rover Challenge (URC)
\item Implemented a planar navigation simulator with A* search, and inverse kinematics for the rover arm
\item Teleoperated the rover during the 2021 virtual URC, in which we placed 3rd globally!
\end{rSubsection}

\end{rSection}


\begin{rSection}{Education}

{\bf University of Washington, Seattle} \hfill {January 2020 - September 2021} \\
M.S. in Computer Science \& Engineering; Advisor: Dr. Byron Boots

\end{rSection}

\begin{rSection}{Academic Publications}

{\bf N. Hatch} and B. Boots. ``The Value of Planning for Infinite-Horizon Model Predictive Control.''
{\em ICRA 2021.}
\sref{https://arxiv.org/abs/2104.02863}.

A. Shaban, C. Cheng, {\bf N. Hatch}, and B. Boots. ``Truncated Back-Propagation for Bilevel Optimization.''
{\em AISTATS 2019.}
\sref{http://proceedings.mlr.press/v89/shaban19a.html}.

\end{rSection}

\begin{rSection}{Technical Strengths}
Python, C++, ROS, Git, PyTorch, TorchScript
\end{rSection}

\end{document}
