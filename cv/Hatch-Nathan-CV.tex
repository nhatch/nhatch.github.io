%%%%%%%%%%%%%%%%%%%%%%%%%%%%%%%%%%%%%%%%%
% Medium Length Professional CV
% LaTeX Template
% Version 2.0 (8/5/13)
%
% This template has been downloaded from:
% http://www.LaTeXTemplates.com
%
% Original author:
% Trey Hunner (http://www.treyhunner.com/)
%
% Important note:
% This template requires the resume.cls file to be in the same directory as the
% .tex file. The resume.cls file provides the resume style used for structuring the
% document.
%
%%%%%%%%%%%%%%%%%%%%%%%%%%%%%%%%%%%%%%%%%

\documentclass{resume} % Use the custom resume.cls style

\usepackage[left=0.75in,top=0.6in,right=0.75in,bottom=0.6in]{geometry} % Document margins

\name{Nathan T. Hatch} % Your name
%\address{123 Broadway \\ City, State 12345} % Your address
\address{\href{https://nhatch.github.io}{https://nhatch.github.io} \\ Seattle, WA, USA \\
         \href{mailto:nhatch2REMOVE_THIS_PART@uw.edu}{nhatch2@uw.edu} \\ +1 970-297-8081 }

\begin{document}

\begin{rSection}{Education}

{\bf University of Washington, Seattle} \hfill {January 2020 - expected 2023} \\
Ph.D. in Computer Science \& Engineering, Advisor: Dr. Byron Boots

{\bf Georgia Institute of Technology} \hfill {August 2017 - December 2019} \\
Ph.D. student in Machine Learning, Advisor: Dr. Byron Boots

{\bf University of Chicago} \hfill {June 2014} \\
B.S. in Mathematics with honors \\
B.S. in Computer Science with honors \\

\end{rSection}

\begin{rSection}{Publications}

A. Shaban, C. Cheng, {\bf N. Hatch}, and B. Boots. {\em Truncated Back-Propagation for Bilevel Optimization.}
Proceedings of the 22nd International Conference on Artificial Intelligence and Statistics (AISTATS 2019).
\href{https://arxiv.org/abs/1810.10667}{https://arxiv.org/abs/1810.10667}.

{\bf N. Hatch}. {\em Group Theory: An Introduction and an Application}. University of Chicago VIGRE REU; 2011.
\href{http://www.math.uchicago.edu/~may/VIGRE/VIGRE2011/REUPapers/Hatch.pdf}{http://www.math.uchicago.edu/~may/VIGRE/VIGRE2011/REUPapers/Hatch.pdf}.

\end{rSection}

\begin{rSection}{Unpublished Research Projects}

\begin{rProject}{High-speed obstacle avoidance for autonomous vehicles}{June 2019 - present}
\item foobar
\end{rProject}

\begin{rProject}{Curriculum-based learning to walk across stepping stones}{May 2018 - May 2019}
\item Mostly on my own, review locomotion literature and invent a new ML-based approach
\item Use the DART simulator to implement and iterate on the algorithm
\item Gave a two-hour presentation summarizing prior work and explaining my current approach
\end{rProject}

\end{rSection}

\begin{rSection}{Work Experience}
\begin{rSubsection}{ Software engineer}{ June 2014 - July 2017}{ eSpark Learning}{ Chicago, IL}
\item Full-stack software engineering in a fast-paced startup environment
\item Led the annual iOS app release, removing 300ms tap delay and rewriting the video uploader
\item Increased sales pipeline by 125\% by integrating our product with Airwatch
\item Improved academic fidelity metric from 80\% to 87\% by refining our app deployment system
\item Implemented Apple's ``Device Assignment'' protocol, making our MDM first-to-market (solo project)
\item Conducted $\sim$20 interviews and code challenge reviews for recruiting
\end{rSubsection}
\begin{rSubsection}{ chiTCP developer}{ October 2013 - June 2014}{ Dept. Computer Science, University of Chicago}{ Chicago, IL}
\item Implemented a TCP-over-TCP daemon for use in Borja Sotomayor's networks class
\item Earned the position based on outstanding performance in that class the previous year
\end{rSubsection}
\begin{rSubsection}{ Software intern}{ June - August 2013}{ Mission Street Manufacturing}{ Santa Barbara, CA}
\item Developed prototype front- and back-end software for consumer-friendly 3D printing
\end{rSubsection}
\end{rSection}

\begin{rSection}{Class Projects}
N. Hatch and E. Wijmans. {\em Probabilistic Graphical Modeling of Data-Dependent Annotator Accuracy for Active Learning.} CS 8803 Probabilistic Graphical Models; Spring 2018. \\
Paper: \sref{https://nhatch.github.io/files/Hatch\_Wijmans\_final\_report.pdf} \\
Slides: \sref{https://nhatch.github.io/files/Hatch\_Wijmans\_presentation\_slides\_v2.pdf}

N. Hatch, A. Sundaresan, M. Dutreix, R. Kuppan, and P. Pattanashetty. {\em Google Landmark Recognition and Retrieval Challenges.} ECE 6254 Statistical Machine Learning; Spring 2018. \\
Paper: \sref{https://nhatch.github.io/files/landmarks\_report.pdf} \\
Poster: \sref{https://nhatch.github.io/files/landmarks\_poster.pdf}

N. Hatch. {\em Unsupervised Curriculum Learning for Image Clustering.} CS 7643 Deep Learning; Fall 2017. \\
Poster: \sref{https://nhatch.github.io/files/image-clustering.pdf}
\end{rSection}

\begin{rSection}{Awards and Honors}
Georgia Institute of Technology, Presidential Fellowship \hfill 2017 – 2021 \\
University of Chicago, Dean's List \hfill 2010 – 2014 \\
University of Chicago, University Scholarship \hfill 2010 – 2014 \\
University of Chicago, National Merit Scholarship \hfill 2010 – 2014 \\
University of Chicago, Student Marshal \hfill 2013 \\
Phi Beta Kappa \hfill 2013 \\
University of Chicago, Fulton Prize for Orchestral Excellence \hfill 2012
\end{rSection}

\begin{rSection}{Technical Strengths}

\begin{tabular}{ @{} >{\bfseries}l @{\hspace{6ex}} l }
Computer Languages & Python, C++, Ruby, Javascript/HTML/CSS \\
Robotics and Simulation Software & ROS, Gazebo, DART \\
Deep Learning Frameworks & PyTorch, TensorFlow \\
Tools & Git, Vim, LaTeX \\
Foreign Languages & Spanish
\end{tabular}

\end{rSection}

\begin{rSection}{Teaching Experience}
\begin{rSubsection}{ Volunteer tutor}{ January 2015 - May 2017}{ Insight Tutoring}{ Chicago, IL}
\item Reviewed homework and class material for three economically disadvantaged sixth-grade students
\item Periodically revisited old material for spaced retrieval practice
\end{rSubsection}
\begin{rSubsection}{ Homework grader}{ October - December 2012}{ Dept. Computer Science, U. of Chicago}{ Chicago, IL}
\item Graded twice-weekly discrete math problem sets for Laszlo Babai's graduate-level class
\end{rSubsection}
\begin{rSubsection}{ SESAME teaching assistant}{ July 2012}{ Dept. Mathematics, U. of Chicago}{ Chicago, IL}
\item Assisted teaching a class for middle school math teachers on ``problem-based learning''
\end{rSubsection}
\begin{rSubsection}{ Young Scholars Program Counselor}{ June - July 2012}{ Dept. Mathematics, U. of Chicago}{ Chicago, IL}
\item Tutored four 9th/10th-grade math students in an advanced summer math program
\item Assisted teaching a class in basic computer programming
\item Gave weekly reports on student progress, including helping to write a diagnostic exam
\end{rSubsection}
\end{rSection}

\begin{rSection}{Hobbies}
\begin{tabular}{ @{} >{\bfseries}l @{\hspace{6ex}} l }
Places visited & Spain, United Kingdom, Japan, Argentina, Brazil, Taiwan, Hong Kong,\\
& Singapore, China, Peru, South Africa, Namibia, Botswana, Zimbabwe, Turkey \\
Other interests &  viola performance, rock climbing, go (the board game)
\end{tabular}
\end{rSection}

\end{document}
