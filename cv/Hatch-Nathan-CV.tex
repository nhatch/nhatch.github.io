%%%%%%%%%%%%%%%%%%%%%%%%%%%%%%%%%%%%%%%%%
% Medium Length Professional CV
% LaTeX Template
% Version 2.0 (8/5/13)
%
% This template has been downloaded from:
% http://www.LaTeXTemplates.com
%
% Original author:
% Trey Hunner (http://www.treyhunner.com/)
%
% Important note:
% This template requires the resume.cls file to be in the same directory as the
% .tex file. The resume.cls file provides the resume style used for structuring the
% document.
%
%%%%%%%%%%%%%%%%%%%%%%%%%%%%%%%%%%%%%%%%%

\documentclass{resume} % Use the custom resume.cls style

\usepackage[left=0.75in,top=0.6in,right=0.75in,bottom=0.6in]{geometry} % Document margins

\name{Nathan T. Hatch} % Your name
\address{\href{https://nhatch.github.io}{https://nhatch.github.io} \\ Seattle, WA, USA \\
         \href{mailto:nhatch2REMOVE_THIS_PART@uw.edu}{nhatch2@uw.edu} \\ +1 970-297-8081 }

\begin{document}

\begin{rSection}{Education}

{\bf University of Washington, Seattle} \hfill {January 2020 - September 2021} \\
M.S. in Computer Science \& Engineering; Advisor: Dr. Byron Boots

{\bf Georgia Institute of Technology} \hfill {August 2017 - December 2019} \\
Ph.D. student in Machine Learning; Advisor: Dr. Byron Boots

{\bf University of Chicago} \hfill {September 2010 - June 2014} \\
B.S. in Mathematics with honors \\
B.S. in Computer Science with honors

\end{rSection}

\begin{rSection}{Publications}

{\bf N. Hatch} and B. Boots. ``The Value of Planning for Infinite-Horizon Model Predictive Control.''
{\em Proceedings of the 2021 International Conference on Robotics and Automation (ICRA 2021).}
\sref{https://arxiv.org/abs/2104.02863}.

A. Shaban, C. Cheng, {\bf N. Hatch}, and B. Boots. ``Truncated Back-Propagation for Bilevel Optimization.''
{\em Proceedings of the 22nd International Conference on Artificial Intelligence and Statistics (AISTATS 2019).}
\sref{http://proceedings.mlr.press/v89/shaban19a.html}.

{\bf N. Hatch}. ``Group Theory: An Introduction and an Application.'' {\em University of Chicago VIGRE REU}; 2011.
\href{http://www.math.uchicago.edu/~may/VIGRE/VIGRE2011/REUPapers/Hatch.pdf}{http://www.math.uchicago.edu/$\sim$may/VIGRE/VIGRE2011/REUPapers/Hatch.pdf}.

\end{rSection}

\begin{rSection}{Unpublished Research Projects}

\begin{rProject}{High-speed off-road autonomous navigation}{June 2019 - present}
\item Conduct weekly field experiments with a Clearpath Warthog robot (a.k.a. Argo J5 XTR) outfitted with cameras and an Ouster OS2 LIDAR sensor
\item Test and improve our custom perception, planning, and control stack
\item Handle physical and electrical hardware integration for new sensors
\item Coordinate tasks for a team of four PIs and half a dozen graduate and undergraduate students
\item Conduct biweekly progress update meetings with the Army Research Lab
\item Repaired a leaky grease seal with remote support from the manufacturer
\item Adapted model-predictive path-integral control to a goal-seeking navigation and mapping stack
\item Sped up the LIDAR processing pipeline to 10Hz to support 3m/s vehicle velocities
\item Soldered header pins to the motor control circuit board to collect serial debug output
\end{rProject}

\begin{rProject}{Curriculum-based learning for bipedal locomotion over rough terrain}{May 2018 - May 2019}
\item Extensively studied prior work in locomotion and summarized it in a two-hour lab presentation
\item Invented an algorithm to learn a real-time, dynamic bipedal locomotion controller
\item Tested the algorithm on challenging ``stepping stone'' environments using the DART simulator
\end{rProject}

\end{rSection}

\begin{rSection}{Clubs and Activities}

\begin{rSubsection}{Software Subsystem Lead}{January 2020 - present}{Husky Robotics Club, U. of Washington}{ Seattle, WA}
\item Write software for teleoperation and autonomous control of a student-designed and -built Mars rover
\item Delegate tasks to maximize our team's chances of winning the University Rover Challenge
\item Recruit software team members by writing an application form and reviewing responses
\item Prepare the software portions of official competition essay and video submissions
\item Coordinate with hardware and electrical subsystem leads to mount sensors and motor boards
\item Implemented a planar navigation simulator with A* search, and inverse kinematics for the rover arm
\end{rSubsection}

\end{rSection}

\begin{rSection}{Personal Projects}
  \begin{tabular}{ll}
    \sref{https://github.com/nhatch/slam} & Factor graph SLAM implemented from scratch \\
    \sref{https://github.com/nhatch/rrt} & RRT motion planning implemented from scratch \\
    \sref{https://github.com/nhatch/ilqr} & Iterative LQR control implemented from scratch \\
    \sref{https://github.com/nhatch/mcts} & A (pretty good) AI for Mancala using Monte Carlo tree search \\
    \sref{https://github.com/nhatch/mnist} & Multilayer perceptron for MNIST digit recognition from scratch \\
    % Removed to help pagebreaks line up with section breaks
    %\sref{https://github.com/nhatch/casia} & Convolutional network for CASIA Chinese characters using Keras
  \end{tabular}
\end{rSection}

\begin{rSection}{Work Experience}

\begin{rSubsection}{Full-stack software engineer}{ June 2014 - July 2017}{ eSpark Learning}{ Chicago, IL}
\item Led the annual iOS app release, removing 300ms tap delay and rewriting the video uploader
\item Increased sales pipeline by 25\% by integrating our product with Airwatch
\item Improved academic fidelity metric from 80\% to 87\% by refining our app deployment system
\item Implemented Apple's ``Device Assignment'' protocol, making our MDM first-to-market (solo project)
\item Conducted $\sim$20 interviews and code challenge reviews for recruiting
\end{rSubsection}

\begin{rSubsection}{ chiTCP developer}{ October 2013 - June 2014}{ Dept. Computer Science, University of Chicago}{ Chicago, IL}
\item Implemented a TCP-over-TCP daemon for use in Borja Sotomayor's networks class
\end{rSubsection}

\begin{rSubsection}{ Software intern}{ June - August 2013}{ Mission Street Manufacturing}{ Santa Barbara, CA}
\item Developed prototype front- and back-end software for consumer-friendly 3D printing
\end{rSubsection}

\end{rSection}

\begin{rSection}{Class Projects}

A. Fishman, N. Hatch, and Y. Yang. {\em Navigating Holiday Traffic.} CSE 599 Reinforcement Learning; Autumn 2020. \\
Paper: \sref{https://nhatch.github.io/files/Navigating\_Holiday\_Traffic.pdf} \\
Slides: \sref{https://docs.google.com/presentation/d/14OqVKhnbL5BtnwXD-2FRr80QebtkZjkICApYpljAoGg}

N. Hatch, G. Parpart, D. Starikov. {\em Deep Robot Localization.} CSE 571 AI-Based Mobile Robotics; Spring 2020. \\
Paper: \sref{https://nhatch.github.io/files/Deep-Robot-Localization.pdf} \\
Slides: \sref{https://docs.google.com/presentation/d/1KFFPJaoL5LctJbqo79rDn1qVgsLjKHGr5tkprjcoeVQ}

A. Baughan, N. Hatch, V. Raganeni, and B. Yang. {\em Search-Based Testing for Robotic Motion Planning Algorithms.} CSE 503 Software Engineering; Winter 2020. \\
Paper: \sref{https://nhatch.github.io/files/Search\_Based\_Testing.pdf} \\
Slides: \sref{https://docs.google.com/presentation/d/1ER0XtU6asJ3MKk-b1D7mIUPU-lZa2kUanL\_SIbILsW0}

S. Foley, N. Hatch, and A. Beedu. {\em A Global Optimal Solution to Non-Minimal Relative Pose Estimation.} ECE 8823 Convex Optimization; Spring 2019. \\
PDF: \sref{https://nhatch.github.io/files/FoleyHatchBeeduNotes.pdf}

N. Hatch and E. Wijmans. {\em Probabilistic Graphical Modeling of Data-Dependent Annotator Accuracy for Active Learning.} CS 8803 Probabilistic Graphical Models; Spring 2018. \\
Paper: \sref{https://nhatch.github.io/files/Hatch\_Wijmans\_final\_report.pdf} \\
Slides: \sref{https://nhatch.github.io/files/Hatch\_Wijmans\_presentation\_slides\_v2.pdf}

N. Hatch, A. Sundaresan, M. Dutreix, R. Kuppan, and P. Pattanashetty. {\em Google Landmark Recognition and Retrieval Challenges.} ECE 6254 Statistical Machine Learning; Spring 2018. \\
Paper: \sref{https://nhatch.github.io/files/landmarks\_report.pdf} \\
Poster: \sref{https://nhatch.github.io/files/landmarks\_poster.pdf}

N. Hatch. {\em Unsupervised Curriculum Learning for Image Clustering.} CS 7643 Deep Learning; Fall 2017. \\
Poster: \sref{https://nhatch.github.io/files/image-clustering.pdf}

Other graduate-level classes (exam-based): Computer Vision, Linear Systems, Theoretical Statistics, Machine Learning Theory, Mathematical Foundations of Machine Learning
\end{rSection}

\begin{rSection}{Awards and Honors}
Georgia Institute of Technology, Presidential Fellowship \hfill 2017 - 2019 \\
University of Chicago, Dean's List \hfill 2010 - 2014 \\
University of Chicago, University Scholarship \hfill 2010 - 2014 \\
University of Chicago, National Merit Scholarship \hfill 2010 - 2014 \\
University of Chicago, Student Marshal \hfill 2013 \\
Phi Beta Kappa \hfill 2013 \\
University of Chicago, Fulton Prize for Orchestral Excellence \hfill 2012
\end{rSection}

\begin{rSection}{Teaching Experience}

\begin{rSubsection}{Head Teaching Assistant, undergraduate machine learning}{Winter 2021}{Dept. Computer Science and Engineering, U. of Washington}{Seattle, WA}
\item Held weekly recitations, wrote and graded homeworks, and managed 10 TAs for a class of 150 students
\item Dealt with academic misconduct investigations for a dozen students
\end{rSubsection}

\begin{rSubsection}{Teaching Assistant, undergraduate machine learning}{Fall 2019}{College of Computing, Georgia Tech}{Atlanta, GA}
\item Graded homework, held weekly office hours, answered Piazza questions, and wrote the final project
\end{rSubsection}

\begin{rSubsection}{Volunteer tutor}{ January 2015 - May 2017}{ Insight Tutoring}{ Chicago, IL}
\item Reviewed homework and class material for three economically disadvantaged sixth-grade students
\item Periodically revisited old material for spaced retrieval practice
\end{rSubsection}

\begin{rSubsection}{Homework grader, graduate discrete mathematics}{ October - December 2012}{ Dept. Computer Science, U. of Chicago}{ Chicago, IL}
\item Graded twice-weekly problem sets for Laszlo Babai's graduate-level class
\end{rSubsection}

\begin{rSubsection}{ SESAME Teaching Assistant}{ July 2012}{ Dept. Mathematics, U. of Chicago}{ Chicago, IL}
\item Assisted teaching a class for middle school math teachers on ``problem-based learning''
\end{rSubsection}

\begin{rSubsection}{ Young Scholars Program Counselor}{ June - July 2012}{ Dept. Mathematics, U. of Chicago}{ Chicago, IL}
\item Tutored four 9th/10th-grade math students in an advanced summer math program
\item Assisted teaching a class in basic computer programming
\item Gave weekly reports on student progress, including helping to write a diagnostic exam
\end{rSubsection}

\end{rSection}

\begin{rSection}{Professional Service}

\begin{rSubsection}{Organizer for New Grad Orientation}{Fall 2020}{Dept. Computer Science and Engineering, U. of Washington}{Seattle, WA}
\item With two co-organizers, planned two days of orientation activities for new CSE PhD students
\item Updated last year's orientation materials to a virtual format due to the COVID-19 pandemic
\item Coordinated faculty course pitches, logistical information, and a scavenger hunt
\end{rSubsection}

\begin{rSubsection}{Co-creator of Machine Learning Student Seminar}{Fall 2019}{Georgia Tech Machine Learning (ML@GT)}{Atlanta, GA}
\item With one co-organizer, started a new seminar with eight presentations to a 25-student audience
\item Invited presenters, including five faculty lightning talks
\item Organized catering, room reservations, and publicity
\end{rSubsection}

\end{rSection}

\begin{rSection}{Technical Strengths}
\begin{tabular}{ @{} >{\bfseries}l @{\hspace{6ex}} l }
Programming Languages & Python, C++, Javascript/HTML/CSS, Ruby \\
Robotics and Simulation Software & ROS, Gazebo, DART \\
Deep Learning Frameworks & PyTorch, TensorFlow \\
Tools & Git, Vim, LaTeX \\
Foreign Languages & Spanish
\end{tabular}
\end{rSection}

\begin{rSection}{Hobbies}
\begin{tabular}{ @{} >{\bfseries}l @{\hspace{6ex}} l }
Places visited & Spain, United Kingdom, Japan, Argentina, Brazil, Taiwan, Hong Kong,\\
& Singapore, China, Peru, South Africa, Namibia, Botswana, Zimbabwe, Turkey \\
Other interests &  viola performance, rock climbing, go (the board game)
\end{tabular}
\end{rSection}

\end{document}
